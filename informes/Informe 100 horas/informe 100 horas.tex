%%%%%%%%%%%%%%%%%%%%%%%%%%%%%%%%%%%%%%%%%%%%%%%%%%%%%%%%%%%%%%%%%%%%%%%%%%%
%
% Plantilla para un art�culo en LaTeX en espa�ol.
%
%%%%%%%%%%%%%%%%%%%%%%%%%%%%%%%%%%%%%%%%%%%%%%%%%%%%%%%%%%%%%%%%%%%%%%%%%%%

\documentclass{article}

% Esto es para poder escribir acentos directamente:
\usepackage[latin1]{inputenc}
% Esto es para que el LaTeX sepa que el texto est� en espa�ol:
\usepackage[spanish]{babel}
% Esto es para que se pueda crear el indice
\usepackage{makeidx}
% Paquetes de la AMS:
\usepackage{amsmath, amsthm, amsfonts}
% para poder poner una hoja apaisada
\usepackage{lscape}
% para poder modificar los tama�os de las imagenes y tablas
\usepackage{graphicx}

% Paquetes de la AMS:
\usepackage{amsmath, amsthm, amsfonts}

% Teoremas
%--------------------------------------------------------------------------
\newtheorem{thm}{Teorema}[section]
\newtheorem{cor}[thm]{Corolario}
\newtheorem{lem}[thm]{Lema}
\newtheorem{prop}[thm]{Proposici�n}
\theoremstyle{definition}
\newtheorem{defn}[thm]{Definici�n}
\theoremstyle{remark}
\newtheorem{rem}[thm]{Observaci�n}

% Atajos.
% Se pueden definir comandos nuevos para acortar cosas que se usan
% frecuentemente. Como ejemplo, aqu� se definen la R y la Z dobles que
% suelen representar a los conjuntos de n�meros reales y enteros.
%--------------------------------------------------------------------------

\def\RR{\mathbb{R}}
\def\ZZ{\mathbb{Z}}

% De la misma forma se pueden definir comandos con argumentos. Por
% ejemplo, aqu� definimos un comando para escribir el valor absoluto
% de algo m�s f�cilmente.
%--------------------------------------------------------------------------
\newcommand{\abs}[1]{\left\vert#1\right\vert}

% Operadores.
% Los operadores nuevos deben definirse como tales para que aparezcan
% correctamente. Como ejemplo definimos en jacobiano:
%--------------------------------------------------------------------------
\DeclareMathOperator{\Jac}{Jac}

%--------------------------------------------------------------------------
\title{\underline{Practica Profesional Supervisada} \\
\large \underline{Informe de las 100 horas} \\
\huge \textbf{ \\ Plataforma concentradora de sensores y eventos digitales} \\ }
\author{Autores: Ignacio Sambataro, Luciano Mantovani\\ \\
  \large Tutor: PhD. Ing. Orlando Micolini \\
  \large Supervisor: Ing. Maximiliano Eschoyez \\ \\
  \small Laboratorio de Arquitectura de Computadoras\\
  \small Facultad de Ciencias Exactas, Fisicas y Naturales\\
  \small Universidad Nacional de Cordoba\\
  \date{A�o 2015}
}


\makeindex
\begin{document}
\maketitle

%resumen
\abstract{Este informe contiene un resumen de las actividades realizadas para la practica profesional
supervisada en las primeras 100 horas de trabajo. El proyecto a realizar es una plataforma concentradora
de sensores y eventos digitales}

%salto de pagina
\clearpage

\addcontentsline{toc}{section}{�ndice} % para que lo a�ada al �ndice de contenidos
\printindex % para que ponga el �ndice aqu�

\section{Introducci�n}
Esta practica esta orientada al dise�o y la construccion de un sistema embebido que concentre las se�ales de varios sensores y varias fuentes de eventos digitales. La idea es que un sistema embebido de uso especifico pueda tercerizar la tarea de obtener, convertir y procesar una se�al de uno o varios sensor o una o varias fuente de eventos digitales.

Surgio de la problematica de algunos proyectos dentro del Laboratorio de Arquitectura de Computadoras que compartian el mismo problema. La necesidad de un sistema que genericamente obtenga las se�ales de los sensores y la pueda transmitir al sistema principal, ya convertidas. Ademas de un contador de eventos que no requiera del uso de interrupciones por software.

En este informe cubrimos los avances hechos en las primeras 100 horas de trabajo en el proyecto.


\section{Requerimientos}
\subsection{Principales}
\begin{itemize}
  \item Leer de 4 a 8 se�ales anal�gicas y convertirlas a digital.
  \item Contar eventos con 3 o 4 contadores distintos.
  \item Transmitir los datos digitales a trav�s de un protocolo serial a alguna otra placa o procesador.
\end{itemize}


\subsection{Secundarios}
\begin{itemize}
  \item Lograr el menor consumo posible.
  \item Buscar la mejor inmunidad al ruido, con una distancia de la placa a los sensores de hasta un m�ximo de 10 metros.
  \item Lograr un producto lo m�s peque�o posible.
  \item Lograr un producto programable
\end{itemize}

\section{Investigacion}
La etapa de investigacion consistio en encontrar un microcontrolador que satisfaga la mayor cantidad de requerimientos principales. El sistema entero consiste en interactuar con el nucleo, que es el microcontrolador, por lo que esta etapa requierio de analisis detallado de las opciones con las se contaba. En el cuadro\ref{tabla_micros} se pueden ver los microcontroladores considerados en la etapa de seleccion.

% Table generated by Excel2LaTeX from sheet 'Sheet1'
\clearpage

\begin{landscape}

\begin{table}[!]
\centering
\begin{flushleft}
% Table generated by Excel2LaTeX from sheet 'Sheet1'
\scalebox{0.7}{
\begin{tabular}{|c|c|c|c|c|c|c|c|c|c|c|c|c|}
\hline
Fabricante & Modelo & RAM(K) & canales ADC & Referencia & Resolucion & ganancia & Contadores & low power & puerto serie & Dimension (') &   Pins &   Us\$ \\
\hline
 Intel & 8XC51GB &    256 &      8 &    GND & 8 bits &     no & 3 (16 bits) &     si & salida y entrada RS232 &      N &     32 &      N \\
\hline
Silicon Labs & C8051F352 &    768 &      8 & DIF/GND & 16 bits &   128x & 4 (16 bits) &     si & Smbus/I2c, UART, SPI & 0,35x0,35 &     32 &    2,3 \\
\hline
 Atmel & AT89C5115 &    256 &      8 &    GND & 8/10 bits &     no & 3 (16 bits) &     si & UART (3 modos Full Duplex) & 0,34x0,34 &     32 &     10 \\
\hline
Microchip & PIC18F4550 &     32 &     13 &    GND & 10 bits &     no & 4(8 y 16 bits) &     si & SPI, I2C, UART/USART, USB & 0,47x0,47 &     44 &   5,36 \\
\hline
   Nec & PD78C17 &   1024 &      8 &    GND & 8 bits &     no & 2 (8 bits) &     si & Msbus/I2c & 0,92x0,70 &     64 &      N \\
\hline
 Maxim & DS4830 & 1024x16 &     16 & DIF/GND & 13 bits &     no & 2(16 bits) &     si & SPI, I2C & 0,2x0,2 &     40 &    7,5 \\
\hline
   NXP & LPC1110 &      4 &      8 &    GND & 10 bits &     no & 2(32 bits) &     si & I2C, UART, Soporte RS-485 & 0,42x0,51 &     20 &    2,5 \\
\hline
 Atmel & ATSAM3A8C &    256 &     16 & DIF/GND & 12 bits &     no & 9(32 bits) &     si & USB, SPI & 0,63x0,63 &     63 &    2,4 \\
\hline
 Atmel & ATSAM3S1A &     64 &      8 & DIF/GND & 10/12 bits & 1x,2x,4x & 3(16 bits) &     si & USB, I2C, SPI & 0,35x0,35 &     44 &    2,5 \\
\hline
 Atmel & ATSAM3S1C &     16 &     16 & DIF/GND & 10/12 bits & 1x,2x,4x & 6(16 bits) &     si & USB, I2C, SPI & 0,63x0,63 &     74 &    2,5 \\
\hline
 Atmel & ATSAMD21J &    256 &     20 & DIF/GND & 12 bits &    16x & 5 (16 bits) &     si & 1 USB 2.0 + 6 I2C/USART/SPI & 0,47x0,47 &     64 &      3 \\
\hline
 Atmel & ATSAMD21G &    256 &     14 & DIF/GND & 12 bits &    16x & 3 (16 bits) &     si & 1 USB 2.0 + 6 I2C/USART/SPI & 0,35x0,35 &     48 &    2,5 \\
\hline
 Atmel & ATSAMD21E &    256 &     10 & DIF/GND & 12 bits &    16x & 3 (16 bits) &     si & 1 USB 2.0 + 4 I2C/USART/SPI & 0,35x0,35 &     32 &    2,5 \\
\hline
Texas Instr & MSP430F5340 &     64 &      9 &    GND & 12 bits &     2x & 7 (distintas) &     si & SPI, I2C, UART & 0,3x0,3 &     48 &    3,3 \\
\hline
    ST & STM32F373CX &    256 &      4 &    GND & 12, 16 bits &    32x & 17 (distintas) &     si & 2 I2C, 3 SIP, 3 USART, 1 USB & 0,35x0,35 &     48 &    2,5 \\
\hline
Atmel AVR & ATmega128 &    128 & 8 (2 c/gain) & 7 DIF, 8 GND & 10 bits & 1x, 10x, 200x & 4 (8 y 16) &     si & USART, SPI & 0,6x0,6 &     64 &      8 \\
\hline
\end{tabular}



}
\end{flushleft}
  \caption{Microcontroladores considerados}\label{tabla_micros}
\end{table}

\end{landscape}


En el momento, habia en el laboratorio una placa de desarrollo de Silicon Labs con el microcontrolador C8051F352. Este mismo fue considerado dentro de las elecciones posibles, como puede verse en el cuadro \ref{tabla_micros}.

Ademas de la ventaja de tenerlo en el mismo laboratorio, la placa de Silicon Labs tiene la particularidad de tener una buena ganancia maxima (128x) a la entrada del ADC, lo cual lo distingue del resto de los microcontroladores analizados. Ademas de esto, cumple con el resto de los requisitos propuestos por nuestro tutor, por lo que consistia en una buena eleccion.

Habiendo hecho este analisis, se decidio optar por utilizar el microcontrolador C8051F352 de Silicon Labs

\begin{equation}\label{eq:area}
  S = \pi r^2
\end{equation}
Uno puede referirse a ecuaciones as�: ver ecuaci�n (\ref{eq:area}).
Tambi�n se pueden mencionar secciones de la misma forma: ver secci�n
\ref{sec:nada}. O citar algo de la bibliograf�a: \cite{Cd94}.

\subsection{Subsection}\label{sec:nada}

M�s texto.

\subsubsection{Subsubsection}\label{sec:nada2}

M�s texto.

% Bibliograf�a.
%-----------------------------------------------------------------
\begin{thebibliography}{99}

\bibitem{Cd94} Autor, \emph{T�tulo}, Revista/Editor, (a�o)

\end{thebibliography}

\end{document} 