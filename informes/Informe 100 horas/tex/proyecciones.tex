\section{Proyeccion a futuro} % (fold)
\label{sec:proyeccion_a_futuro}

En las proximas 100 horas de trabajo, el objetivo es finalizar la idea de los dise\~nos propuestos hasta el momento, corregirlos si es necesario, y luego implementar hasta donde se llegue. Si es posible, llegar a tener pruebas de laboratorio funcionando. Probar el dispositivo conectandolo con uno o varios sensores que provean datos crudos, y brindarle la informacion ordenada a travez del protocolo UART \ref{ssub:interfaz_serial} a algun servidor activo que la almacene en una base de datos.

En concreto, los objetivos son los siguientes:
\begin{itemize}
	\item Completar el dise\~no de hardware
	\item Impresion del circuito en placa de cobre
	\item Ampliar la lista de comandos de la interfaz MML para admitir mayor programabilidad a la plataforma
	\item A\~nadir mas contadores de eventos utilizando las funcionalidades de Timer y de PCA del microcontrolador utilizado
	\item A\~nadir la funcionalidad de especificar una frecuencia deseada de muestreo en la adquisicion de datos del conversor analogico-digital
	\item Realizar un prototipo del hardware y probarlo con el software desarrollado
	\item Conformar una base de testing que permite probar que todas las funcionalidades de hardware y software funcionen como se espera
\end{itemize}


% section proyeccion_a_futuro (end)