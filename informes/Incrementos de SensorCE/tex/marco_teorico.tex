\section{Funcionamiento del sensor} % (fold)
\label{sec:funcionamiento_del_sensor}
el sensor es una estructura metalica compuesta por un motor que hace girar unas aspas que se se encargan de blindar y desblindar una placa que a su vez se carga y descarga con la electricidad estatica del ambiente. esta carga y descarga continua es lo que justamente se termina transformando en el nivel de voltaje que nos indica el nivel de electricidad estatica del ambiente, que es lo que queremos saber. Si imaginamos a la placa y a las aspas como un capacitor que se blinda y desblinda es mas facil. En el momento que las aspas estan descuburiendo la placa, el capacitor se carga, y en el momento que se cubre la placa, el capacitor se descarga. La descarga se hace sobre un amplificador que luego va a un conversor analogico digital, que termina en la lectura de un valor que nos dice el nivel del campo electrostatico ambiental.



\section{Marco Teorico} % (fold)
\label{sec:marco_teorico}

% The strength of the electric field could be measured, in principle, by placing a volt meter across plates placed some distance apart. However, because the meter will have some input impedance (10 Meg typically), any voltage induced on the plates will quickly drain away, and would not be useful for measuring the static field. To make measurements of the static field, the chopper technique is used. The chopper blade is arranged over the Sense Plate and rotated so that it periodically shields, and exposes the Sense Plate to the electric field. To properly do this, the Rotor must be grounded. The Sense Plate is grounded through a transconductance amplifier, which converts the Sense Plate's ground current to a voltage. As the Sense Plate is exposed to the Field, the field induces ground currents as it attracts or repels charge from the Sense Plate. As the plate is shielded from the field, the induced charge drains away. So the chopper induces an AC ground current which is proportional to electric field strength. 
% This AC signal could then be rectified to drive a DC volt meter or be plotted on a scope. However, by doing this only the magnitude of the field, not the sign (positive of negative) would be measured. Also, any noise in the signal would also affect the output. The signal conditioning for this Field Mill uses a synchronous demodulation technique to preserve field sign information and reduce noise. 

% It works like this: The blade position is measured using an LED and photo transistor. The Position Sensor clock signal is used to effectively amplify the AC signal from the Sense Plate amplifier by either +1 or -1, depending on Rotor blade position. This has the effect of synchronously rectifying the AC signal, preserving sign. This rectified signal is then low pass filtered to remove ripple. Alternatively, this circuit function can be thought of as the mixing of two identical frequencies, resulting in output with frequency content at DC, and twice the input frequency. The low pass filter then passes only the DC component. This line of thought will also show how noise at frequencies other that the position clock frequency are rejected. 

La intensidad de un campo electrico se puede medir, en principio, colcando un medidor de voltaje entre dos placas metalicas paralelas separadas por una distancia. El problema de esto es que, como el medidor de voltaje suele tener una impedancia alta en la entrada, cualquier voltaje inducido en las placas se pierde rapidamente, y no podria usarse para medir el capo electrico. Para arreglar esto, se utiliza la tecnica de las aspas. Se coloca una placa conductora, y sobre la misma se posiciona un sistema con aspas de forma que cuando estas roten, se cubra y se exponga periodicamente la placa conductora al campo electrico ambiental. Para lograr esto apropiadamente, el rotor que hace girar las aspas debe estar conectado a tierra. La placa conductora esta conectada a tierra a traves de un amplificador de transconductancia, que convierte la corriente que va desde la placa a tierra en una tension. A medida que la placa conductora este expuesta al campo electrico, el campo induce una corriente a tierra mientras que atrae o repele la carga de la placa conductora. A medida que la placa esta cubierta del campo electrico, la carga inducida se drena. Entonces las placas inducen una corriente alterna a masa que es proporcional a la intensidad del campo electrico estatico. Esta corriente alterna luego puede ser rectificada para utilizarla como entrada a un conversor analogico digital y obtener asi la intensidad del campo electrico medido.
Una vez obtenida la intensidad, es necesario saber el signo del 




% section marco_teorico (end)