%--------------------------------------------------------------
%     Ejemplo de un artículo con LaTeX
%--------------------------------------------------------------
\documentclass[a4paper]{article}
\usepackage[latin1]{inputenc}
\usepackage[spanish]{babel}
\usepackage{amssymb}
\usepackage{amsthm}
\usepackage{amsmath}

%para usar hipertexto en las referencias
\usepackage{hyperref}
\hypersetup{
    colorlinks,
    citecolor=black,
    filecolor=black,
    linkcolor=black,
    urlcolor=black
}
%para incluir graficos
\usepackage{graphicx}
%para permitir que la imagen este donde yo quiero y no en otro lugar
\usepackage{float}

\usepackage{fancyhdr}

\pagestyle{fancy}
\fancyhf{}
\rhead{Año 2015}
\lhead{Proyecto Integrador}
\rfoot{Pagina \thepage}


%--------------------------------------------------------------
\title{\underline{Proyecto integrador} \\
\large \underline{Primer incremento} \\
\huge \textbf{ \\ Sistema inteligente de deteccion de campo electrostatico atmosferico} \\ }
\author{Autores: Ignacio Sambataro, Luciano Mantovani\\ \\
  \large Tutor: PhD. Ing. Orlando Micolini \\
  \large Supervisor: Ing. Maximiliano Eschoyez \\ \\
  \small Laboratorio de Arquitectura de Computadoras\\
  \small Facultad de Ciencias Exactas, Fisicas y Naturales\\
  \small Universidad Nacional de Cordoba\\
  \date{Año 2015}
}
%--------------------------------------------------------------
\begin{document}
\maketitle
\thispagestyle{empty}

\clearpage

\tableofcontents
\thispagestyle{empty}

\clearpage
\setcounter{page}{1}

%--------------------------------------------------------------

\section{Introducción} % (fold)
\label{sec:introduccion}

Esta práctica fue orientada al diseño y la construcción de un sistema embebido que concentre las señales de varios sensores y varias fuentes de eventos digitales. La idea es que un sistema embebido de uso específico pueda tercerizar la tarea de obtener, convertir y procesar una señal de uno o varios sensor o una o varias fuente de eventos digitales.

Este proyecto surgió de la problemática de algunos trabajos realizados dentro del Laboratorio de Arquitectura de Computadoras que compartían el mismo inconveniente. La necesidad de un sistema que genéricamente obtenga las señales de los sensores y la pueda transmitir al sistema principal, ya convertidas. Además de un contador de eventos que no requiera del uso de interrupciones por software.

En este informe cubrimos los avances hechos en 200 horas de trabajo en el proyecto.
% section introducción (end)




\section{haciendo andar el sensor} % (fold)
\label{sec:haciendo_andar_el_sensor}
lo primero que hicimos fue darle arranque al sensor, digamos para ver que onda. hablamos con la gente del famaf que lo habia hecho para ver como arrancarlo. cuando lo pudimos hacer arrancar hicimos un par de pruebas con el osciloscopio para ver como variaba el campo electromagnetico al acercarle una regla cargada con electricidad estatica del pelo.

A todo esto nos explicaron como realmente funciona el sensor.

% section haciendo_andar_el_sensor (end)



\section{funcionamiento del sensor} % (fold)
\label{sec:funcionamiento_del_sensor}
el sensor es una estructura metalica compuesta por un motor que hace girar unas aspas que se se encargan de blindar y desblindar una placa que a su vez se carga y descarga con la electricidad estatica del ambiente. esta carga y descarga continua es lo que justamente se termina transformando en el nivel de voltaje que nos indica el nivel de electricidad estatica del ambiente, que es lo que queremos saber. Si imaginamos a la placa y a las aspas como un capacitor que se blinda y desblinda es mas facil. En el momento que las aspas estan descuburiendo la placa, el capacitor se carga, y en el momento que se cubre la placa, el capacitor se descarga. La descarga se hace sobre un amplificador que luego va a un conversor analogico digital, que termina en la lectura de un valor que nos dice el nivel del campo electrostatico ambiental.

% section funcionamiento_del_sensor (end)

\section{alimentacion del motor} % (fold)
\label{sec:alimentacion_del_motor}
el motor que mueve las aspas es un motor trifasico de 12V controlado por una señal modulada por ancho de pulso. Esta señal de PWM, en el momento que nos dieron el sensor, estaba siendo generada por un circuito que generaba mucha inestabilidad en la velocidad del motor. la estabilidad del motor es importante ya lo explico mas adelante porque. Asi que en principio uno de los objetivos fue generar el PWM desde otro lugar que no sea ese circuito. otro de los objetivos era agarrar la señal del sensor y meterla a la plataforma nuestra y ver si media. asi que en paralelo fuimos haciendo las dos.

% section alimentacion_del_motor (end)

\section*{paso desde el 555 al arduino}
\section{paso desde el 555 al arduino} % (fold)
\label{sec:paso_desde_el_555_al_arduino}

como el emi y diego habian hecho un programa para hacer andar un motor igual para la helice del dron, le sacamos el programa de arduino que lo hacia andar y se lo enchufamos al nuestro. despues de renegar un ratito anduvo a pleno. un poco aprendimos como es la secuencia de los distintos anchos de pulso que tiene que recibir el motor para salir arrancando bien. estubo andando en el arduino un rato hasta que pudimos hacer la transferencia completa desde el programa de arduino al de la silicon labs que tiene un modulo especifico de PWM que venia joya para lo que estabamos haciendo. Arduino en realidad tambien tiene un modulo de pwm pero no estbamos trabajando en arduino para hacer las funciones que le competian al motor directamente y ademas la funcion que tenian los pibes generaba los anchos de pulso usando interrupciones, y no con el pwm. Habia que hacer una transicion de codigo y ya que estabamos pasamos directamente a la silicon labs para asi tambien ahorrar en placas.
% section paso_desde_el_555_al_arduino (end)

\section*{conteo de rpm del motor}
\section{conteo de rpm del motor} % (fold)
\label{sec:conteo_de_rpm_del_motor}

el eje del motor, ademas de hacer girar las aspas, hace girar otro grupo de 4 aspas que hacen interferir la señal de un optoacoplador. la onda es que cada vez que pasa el aspa, se tapa la luz que va desde el led al transistor, haciendo que haya una señal cuadrada en la salida. con esta señal cuadrada puedo saber la velocidad de giro del motor. y como es importante que se mantenga estable, puedo establecer una velocidad ideal, y usando la velocidad real del motor, ajustarla para que tienda a ser la real, y asi establecer el motor. Cada 4 flancos de subida de la señal del optoacoplador, tendriamos una vuelta entera del motor.
% section conteo_de_rpm_del_motor (end)


\section*{estabilidad del motor}
lo ideal era hacer un PID para estabilizar el motor, pero no se hizo el calculo porque era un viaje sacar la funcion de transferencia de todas las partes que sumadas dan la velocidad del motor. No solo esta el motor sino tambien la estructura entera, teniendo en cuenta que esta puesta sobre un eje entero. habia que hacer un monton de calculos teniendo en cuenta la estructura y ademas tambien saber la funcion de transferencia del driver PWM-Trifasico.
	Para simplificar, lo que hice fue hacer un control de velocidad por software. Haciendo calculos nos dimos cuenta
que para una velocidad ideal de 2800 RPM, se necesita que haya 18 flancos en 100ms. Porque elegimos 100ms? basicamente lo que se tiene que hacer cuando se hace una correccion de este tipo es establecer una base de tiempo en la que vos vas a tomar datos y despues con esos datos corregis. mientras menor es la base de tiempo, menos datos podes tomar. al mismo tiempo, mientras mayor es la base de tiempo, hay una menor cantidad de correcciones por unidad de tiempo , que contribuyen a la estabilidad. Es necesario establecer una base de tiempo que no se vaya a ningun extremo. Basandonos en el experimento y prueba y error, decidimos ir con la de 100ms que por ahora no dio ningun problema.

\section*{porque es necesario que el motor se mantenga estable?}
la velocidad de las aspas determinan el tiempo que el capacitor se esta cargando y descargando, por lo que actua directamente sobre la medicion final. Con una electricidad estatica ambiental estable, si fluctua la velocidad del motor, fluctua la medicion, lo cual daria que el campo esta variando cuando en la realidad esta estable. O sea da una lectura incorrecta. Por eso es necesario calibrarlo en una velocidad dada, y tomar muestras del valor del campo segun esa velocidad que debe considerarse como estable. y para que sea estable, es necesario controlarla.

\section*{como se trabaja la señal que sale del sensor??}
el sensor tiene dos cables de salida que dan en crudo los datos de la placa cargada. esa señal es la que hay que convertir, pero antes de convertirla hay que amplificarla con el menor ruido posible para que los datos sean legibles por el ADC de la silicon labs. Basicamente para amplificarla sin ruido lo que se hizo fue poner un par de amplificadores operacionales, uno actuando como adaptador de impedancia y otro como un amplificador de tension no inversor.

El adaptador de impedancia copia a la salida lo que tiene a la entrada, pero eliminando ruido, y el que actua de amplificador, amplifica con una razon que depende de las resistencias con las que se hace operar (complementar con info de internet sobre aplicaciones de los operacionales.)

\section*{como se trabaja la señal que sale del optoacoplador??}
la placa cuenta con un contador de eventos. lo que hace un contador de eventos es basicamente contar cada vez que aparece un flanco de subida en uno de los pines de entrada. Tiene una precision de 16 bits, por lo que cuando llega a 65536 se resetea y vuelve a 0. La señal del optoacoplador va directo a la entrada del contador de eventos.

\section*{que es un PWM?}
La modulacion por ancho de pulso, en ingles \textbf{Pulse Width Modulator}, es una tecnica en la cual se modifica el ciclo de trabajo de una señal de una señal periodica para distintos propositos. Para este caso, el PWM se utiliza como entrada para un driver de un motor trifasico junto con una señal de 12 V que hacen que se mueva el motor.

\section*{arranque del motor con la señal de pwm}
el driver del motor pasa de la señal de un pwm a una señal trifasica que alimenta el motor. Es necesario que para el arranque, se le suministre una secuencia con un ancho de pulso especifico creciente, que provoca que el motor arranque. una vez que arranca ya se puede modificar el ancho de pulso a gusto para que ande a velocidades distintas.

La frecuencia del PWM es de 380Hz, y necesita 3 anchos de pulso distintos en 3 periodos de 600ms cada uno. Luego de eso se le da una velocidad de arranque que corresponde a un ancho de pulso bastante ancho, y despues se baja a uno mas tranqui para darle una velocidad estable. luego de eso se le aplica el control de velocidad que actua directamente sobre el valor del ciclo de trabajo teniendo en cuenta la velocidad medida con el sistema que detecta la velocidad del motor que explique en la otra seccion


\end{document}