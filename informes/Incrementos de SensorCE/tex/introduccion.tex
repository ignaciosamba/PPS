\section{Introduccion} % (fold)
\label{sec:introduccion}
%bueno la idea fue la siguiente. un dia viene orlando y me trae el sensor de estatica y me dice que le gustaria hacerlo andar con la plataforma concentradora que habiamos hecho. la idea era que tomaramos la se\~nal que recibe y guardarla en algun lado. hasta el momento me dio el sensor y un circuito que ni idea que hacia. me explico que el motor que lo hacia andar era un motor trifasico que arrancaba con un tren de pulsos especifico, y que una vez dado el arranque empezaba a medir. no me especifico mucho mas, el tema era hacerlo andar y de ahi veiamos que haciamos.

Este proyecto fue una actividad secundaria dentro del desarrollo del proyecto integrador. Dio la oportunidad de probar la plataforma concentradora de sensores, y al mismo tiempo la provision de un sistema controlador para un sensor de campo electrostatico. 

El sensor utilizado es un dispositivo que mide la intensidad del campo electrico en el ambiente devido al campo electrostatico generado por la carga electrica de las nubes en el momento. Se puede utilizar para detectar la posibilidad de que caiga un rayo en una zona cercana al sensor, y tambien para investigar los efectos de la electricidad estatica. Para funcionar, el sensor necesita de un motor que gire a velocidad constante. Este motor funciona en base a un driver que tiene como entrada una se\~nal modulada por ancho de pulso. Mientras el motor gire a velocidad constante, es posible obtener datos validos del sensor. Tanto el funcionamiento del sensor como la adquisicion de las mediciones fue logrado con funcionalidades de la concentradora de sensores y eventos digitales, mas otras caracteristicas que ofrece el microcontrolador C8051F352, como ser el modulo de PWM para generar la se\~nal modulada por ancho de pulso, y el uso de los timers para establecer las bases de tiempo que se utilizaron para hacer un control de estabilidad en la velocidad del motor.

% section introduccion (end)