\section{Silicon Labs \emph{C8051F352}} % (fold)
\label{sec:silicon_labs_c8051f352}

En esta seccion se explicaran brevemente las funcionalidades que se utilizaron del microcontrolador elegido. Para informacion mas detallada referirse al datasheet del microcontrolador. \cite{bib:datasheet}

\subsection{Conversor Analogico-Digital} % (fold)
\label{ssub:conversor_analogico_digital}

El C8051F350 incluye un ADC Sigma-Delta completamente diferencial de 16 bits con capacidad de calibracion interna, con capacidad de 8 mediciones simultaneas en modo singular, y 4 mediciones simultaneas en modo diferencial. Tiene dos filtros separados de decimacion programables con un throughput de hasta 1KHz. Tiene un voltaje de referencia interno de 2.5 V, y admite la administracion de voltajes de referencia externos. Cada canal puede ser amplificado 1, 2, 4, 8, 16, 32, 64, o 128 veces.

\subsubsection{Modos de medicion} % (fold)
\label{ssub:modos_de_medicion}

\begin{itemize}
  \item \textbf{Singular}: Entrada independiente medida con respecto a GND.
  \item \textbf{Diferencial}: Dos entradas medidas con respecto una de otra.
\end{itemize}
% subsubsection modos_de_medicion (end)

\subsubsection{Modos de conversion} % (fold)
\label{ssub:modos_de_conversion}

\begin{itemize}
  \item \textbf{Singular}: Indica al ADC que genere informacion suficiente para producir un resultado luego de un ciclo de conversion. El filtro puede ser \emph{fast-filter} o \emph{SINC3}. En el primero, el resultado esta disponible luego de un solo ciclo de conversion del ADC. En el segundo, espera 3 ciclos antes de que este disponible el resultado.
  \item \textbf{Continua}: El ADC comienza una nueva conversion inmediatamente despues de terminar la anterior. Los filtros funcionan de manera analoga al modo singular.
\end{itemize}
% subsubsection modos_de_conversion (end)
% subsubsection conversor_analogico_digital (end)

\subsection{Contadores} % (fold)
\label{sub:contadores}

El C8051F350 cuenta con 4 contadores: Timer0, Timer1, Timer2 y Timer3. Timer0 y Timer1 cuentan con 3 modos de operacion:
\begin{itemize}
  \item Contador de 13 bits
  \item Contador de 16 bits
  \item Contador de 8 bits con valor de retorno
\end{itemize}
Timer0 cuenta con un modo mas que es el de separarse en dos contadores independientes de 8 bits.
Timer2 y Timer3 tienen 2 modos:
 \begin{itemize}
   \item Contador de 16 bits con valor de retorno
   \item Dos contadores independientes de 8 bits con valor de retorno
 \end{itemize}

Todos los timers pueden ser clockeados por fuente interna o externa. En el caso de los timers 2 y 3 la fuente externa se divide por 8.
% subsection contadores (end)

\subsection{Interfaz Serial} % (fold)
\label{sub:interfaz_serial}

El C8051F352 cuenta con dos protocolos de transferencia serial de datos: SMBus y UART. \textbf{SMBus} es una interfaz de dos cables bi-direccional. Es compatible con $I^{2}$C, es por esto que se considera que este microcontrolador cuenta con $I^{2}$C por el hecho de tener SMBus. \textbf{UART} es una interfaz serial asincrona full-duplex.
% subsection interfaz_serial (end)

\subsection{Flash} % (fold)
\label{sub:flash}

Incluye una memoria Flash de 8 Kilobytes reprogramable dentro del chip para almacenamiento de datos no volatil. Es programable mediante interfaz C2 o por software
% subsection flash (end)
% section silicon_labs_c8051f352 (end)