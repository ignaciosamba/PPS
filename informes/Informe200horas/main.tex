%%%%%%%%%%%%%%%%%%%%%%%%%%%%%%%%%%%%%%%%%%%%%%%%%%%%%%%%%%%%%%%%%%%%%%%%%%%
%
% Plantilla para un artículo en LaTeX en espa\~nol.
%
%%%%%%%%%%%%%%%%%%%%%%%%%%%%%%%%%%%%%%%%%%%%%%%%%%%%%%%%%%%%%%%%%%%%%%%%%%%

\documentclass[a4paper]{article}
\usepackage[spanish]{babel}
\selectlanguage{spanish}
\usepackage[utf8]{inputenc}
% Paquetes de la AMS:
\usepackage{amsmath, amsthm, amsfonts}
% para poder poner una hoja apaisada
\usepackage{lscape}
%para incluir graficos
\usepackage{graphicx}
%para permitir que la imagen este donde yo quiero y no en otro lugar
\usepackage{float}
% para hipertexto
\usepackage{url}
% para poder envolver imagenes con texto
\usepackage{wrapfig}
%para los apendices
\usepackage[titletoc]{appendix}
% Esto es para poder modificar las dimensiones del texto
\usepackage{geometry}
\geometry{a4paper}
% Esto es para que el [h] me ponga las imagenes dentro de las secciones
\usepackage[section]{placeins}


\graphicspath{ {imagenes/} }

%-------- ENCABEZADO Y PIE DE PAGINA--------
\usepackage{float}

\usepackage{fancyhdr}
 
\pagestyle{fancy}
\fancyhf{}
\rhead{Año 2015}
\lhead{Practica Profesional Supervisada}
\rfoot{Pagina \thepage}
\lfoot{Mantovani - Sambataro}

%-------------------------------------------

%para usar hipertexto en las referencias
\usepackage{hyperref}
\hypersetup{
    colorlinks,
    citecolor=black,
    filecolor=black,
    linkcolor=black,
    urlcolor=black
}


%--------------------------------------------------------------------------
\title{\underline{Practica Profesional Supervisada} \\
\large \underline{Informe de las 200 horas} \\
\huge \textbf{ \\ Plataforma concentradora de sensores y eventos digitales} \\ }
\author{Autores: Ignacio Sambataro, Luciano Mantovani\\ \\
  \large Tutor: PhD. Ing. Orlando Micolini \\
  \large Supervisor: Ing. Maximiliano Eschoyez
 % \small Facultad de Ciencias Exactas, Fisicas y Naturales\\
 % \small Laboratorio de Arquitectura de Computadoras\\
 % \small Universidad Nacional de Cordoba\\
  \date{}
}



%\makeindex
\begin{document}
\maketitle
\begin{figure}[H]
\centering
\includegraphics[width=10cm, height = 3.2cm]{Escudo}
\end{figure}

\begin{center}
\small Facultad de Ciencias Exactas, Fisicas y Naturales \\
\small Laboratorio de Arquitectura de Computadoras \\
\end{center}
\begin{center}
\small \textbf{A\~no 2015} \\
\end{center}



\clearpage
%resumen
\abstract{
%En este informe se describen las primeras 100 horas de trabajo de la Practica Profesional Supervisada. Se comenz\'o un proyecto nuevo, sugerido por el director, Orlando Micolini, y se trabajo todo el tiempo en el Laboratorio de Arquitectura de Computadoras. Se documentaron los requerimientos, la etapa de investigaci\'on, los dise\~nos de hardware y software hasta el momento, y las implementaciones que se llegaron a lograr. Lo que se llego a implementar se hizo en una placa de desarrollo prestada. Las razones de utilizaci\'on de esta placa se explican en el documento. Algunos requerimientos básicos llegaron a cumplirse dentro de las 100 horas, pero solamente prototipos de prueba en la placa de desarrollo. El primer prototipo real tiene la mayor parte del dise\~no de hardware y software hecho, pero todavía no esta implementado. 

En el campo de trabajo de los sistemas embebidos, suele ser necesaria la interacci\'on con sensores de distintos tipos. Las distintas maneras de obtener los datos de los sensores para procesarlos en el sistema final puede llegar a ser una tarea extra para este. 

El objetivo de esta practica fue diseñar y prototipar un sistema que concentre señales de sensores y cuente eventos de fuentes digitales externas; de forma que todo el trabajo de obtenci\'on de señales, digitalizado y procesamiento de las mismas, sea hecho por una placa externa, para que luego se env\'ie por un protocolo serial al sistema final, que termina de procesar los datos de las mediciones. 

El resultado final fue un prototipo de un sistema programable, capaz de tomar señales de 8 canales anal\'ogicos distintos, ya sea en forma diferencial o singular, y contar eventos de dos fuentes de señales digitales distintas. Todos estos datos se env\'ian por protocolo serial hacia el sistema final conectado, en cadenas de caracteres.
}

%salto de pagina
\thispagestyle{empty}
\clearpage

\tableofcontents
\thispagestyle{empty}
\clearpage

\setcounter{page}{1}
%introduccion
\section{Introducción} % (fold)
\label{sec:introduccion}

Esta practica fue orientada al diseño y la construccion de un sistema embebido que concentre las señales de varios sensores y varias fuentes de eventos digitales. La idea es que un sistema embebido de uso especifico pueda tercerizar la tarea de obtener, convertir y procesar una señal de uno o varios sensor o una o varias fuente de eventos digitales.

Surgio de la problematica de algunos proyectos dentro del Laboratorio de Arquitectura de Computadoras que compartian el mismo problema. La necesidad de un sistema que genericamente obtenga las señales de los sensores y la pueda transmitir al sistema principal, ya convertidas. Ademas de un contador de eventos que no requiera del uso de interrupciones por software.

En este informe cubrimos los avances hechos en 200 horas de trabajo en el proyecto.
% section introducción (end)


%requerimientos
\section{Requerimientos}
\subsection{Principales}
\begin{itemize}
  \item Leer de 4 a 8 se\~nales analogicas y convertirlas a digital.
  \item Contar eventos con 3 o 4 contadores distintos.
  \item Transmitir los datos digitales a trav\'es de un protocolo serial a alguna otra placa o procesador.
\end{itemize}


\subsection{Secundarios}
\begin{itemize}
  \item Lograr el menor consumo posible.
  \item Buscar la mejor inmunidad al ruido, con una distancia de la placa a los sensores de hasta un m\'aximo de 10 metros.
  \item Lograr un producto lo más peque\~no posible.
  \item Lograr un producto programable
\end{itemize}

%investigacion
\section{Investigacion}
La etapa de investigacion consistio en encontrar un microcontrolador que satisfaga la mayor cantidad de requerimientos principales. El sistema entero consiste en interactuar con el nucleo, que es el microcontrolador, por lo que esta etapa requierio de analisis detallado de las opciones con las se contaba. En el cuadro\ref{tabla_micros} se pueden ver los microcontroladores considerados en la etapa de seleccion.

% Table generated by Excel2LaTeX from sheet 'Sheet1'
\clearpage

\begin{landscape} % TABLA DE MICROS

\begin{table}[!]
\centering
\begin{flushleft}
% Table generated by Excel2LaTeX from sheet 'Sheet1'
\scalebox{0.7}{
\begin{tabular}{|c|c|c|c|c|c|c|c|c|c|c|c|c|}
\hline
Fabricante & Modelo & RAM(K) & canales ADC & Referencia & Resolucion & ganancia & Contadores & low power & puerto serie & Dimension (') &   Pins &   Us\$ \\
\hline
 Intel & 8XC51GB &    256 &      8 &    GND & 8 bits &     no & 3 (16 bits) &     si & salida y entrada RS232 &      N &     32 &      N \\
\hline
Silicon Labs & C8051F352 &    768 &      8 & DIF/GND & 24 bits &   128x & 4 (16 bits) &     si & Smbus/$I^{2}$C, UART, SPI & 0,35x0,35 &     32 &    2,3 \\
\hline
 Atmel & AT89C5115 &    256 &      8 &    GND & 8/10 bits &     no & 3 (16 bits) &     si & UART (3 modos Full Duplex) & 0,34x0,34 &     32 &     10 \\
\hline
Microchip & PIC18F4550 &     32 &     13 &    GND & 10 bits &     no & 4(8 y 16 bits) &     si & SPI, $I^{2}$C, UART/USART, USB & 0,47x0,47 &     44 &   5,36 \\
\hline
   Nec & PD78C17 &   1024 &      8 &    GND & 8 bits &     no & 2 (8 bits) &     si & Msbus/$I^{2}$C & 0,92x0,70 &     64 &      N \\
\hline
 Maxim & DS4830 & 1024x16 &     16 & DIF/GND & 13 bits &     no & 2(16 bits) &     si & SPI, $I^{2}$C & 0,2x0,2 &     40 &    7,5 \\
\hline
   NXP & LPC1110 &      4 &      8 &    GND & 10 bits &     no & 2(32 bits) &     si & $I^{2}$C, UART, Soporte RS-485 & 0,42x0,51 &     20 &    2,5 \\
\hline
 Atmel & ATSAM3A8C &    256 &     16 & DIF/GND & 12 bits &     no & 9(32 bits) &     si & USB, SPI & 0,63x0,63 &     63 &    2,4 \\
\hline
 Atmel & ATSAM3S1A &     64 &      8 & DIF/GND & 10/12 bits & 1x,2x,4x & 3(16 bits) &     si & USB, $I^{2}$C, SPI & 0,35x0,35 &     44 &    2,5 \\
\hline
 Atmel & ATSAM3S1C &     16 &     16 & DIF/GND & 10/12 bits & 1x,2x,4x & 6(16 bits) &     si & USB, $I^{2}$C, SPI & 0,63x0,63 &     74 &    2,5 \\
\hline
 Atmel & ATSAMD21J &    256 &     20 & DIF/GND & 12 bits &    16x & 5 (16 bits) &     si & 1 USB 2.0 + 6 $I^{2}$C/USART/SPI & 0,47x0,47 &     64 &      3 \\
\hline
 Atmel & ATSAMD21G &    256 &     14 & DIF/GND & 12 bits &    16x & 3 (16 bits) &     si & 1 USB 2.0 + 6 $I^{2}$C/USART/SPI & 0,35x0,35 &     48 &    2,5 \\
\hline
 Atmel & ATSAMD21E &    256 &     10 & DIF/GND & 12 bits &    16x & 3 (16 bits) &     si & 1 USB 2.0 + 4 $I^{2}$C/USART/SPI & 0,35x0,35 &     32 &    2,5 \\
\hline
Texas Instr & MSP430F5340 &     64 &      9 &    GND & 12 bits &     2x & 7 (distintas) &     si & SPI, $I^{2}$C, UART & 0,3x0,3 &     48 &    3,3 \\
\hline
    ST & STM32F373CX &    256 &      4 &    GND & 12, 16 bits &    32x & 17 (distintas) &     si & 2 $I^{2}$C, 3 SIP, 3 USART, 1 USB & 0,35x0,35 &     48 &    2,5 \\
\hline
Atmel AVR & ATmega128 &    128 & 8 (2 c/gain) & 7 DIF, 8 GND & 10 bits & 1x, 10x, 200x & 4 (8 y 16) &     si & USART, SPI & 0,6x0,6 &     64 &      8 \\
\hline
\end{tabular}



}
\end{flushleft}
  \caption{Microcontroladores considerados}\label{tabla_micros}
\end{table}

\end{landscape}

\subsection{Parametros tenidos en cuenta en la seleccion del microcontrolador}
\begin{itemize}
  \item \textbf{RAM:} No es un requisito principal, pero en caso de tener que decidir entre dos micros similares, el tama\~no de la memoria puede ser un factor para tomar la decision final
  \item \textbf{Cantidad de canales del ADC:} Mientras mas canales se tengan, mas se\~nales analogicas de entrada pueden haber, y mas se\~nales de sensores se podran procesar simultaneamente
  \item \textbf{Referencia:} Nos dice si los pines del ADC se pueden usar como entrada diferencial o unicamente con referencia a GND. Esto es porque en el caso que haya 16 pines para el ADC y puedan usarse todos como entrada diferencial, se podrann usar como maximo la mitad de los pines, es decir 8.
  \item \textbf{Resolucion:} Es la cantidad de bits con la que se representa el dato convertido. A mayor resolucion, mayor presicion de la conversion.
  \item \textbf{Ganancia:} Una buena ganancia interna en el micro es necesaria para una amplificacion de la se\~nal.  evitando la mayor cantidad de ruido posible. Este parametro es clave si se quiere trabajar con sensores que funcionan a voltajes muy peque\~nos en ambientes susceptibles al ruido electrico.
  \item \textbf{Contadores:} Cantidad de timers en el microcontrolador que se utilizarian como contadores de eventos(es necesario que puedan ser clockeados por fuente externa, es decir, que la fuente que incrementa el contador provenga de eventos externos y no interiores al microcontrolador).
  \item \textbf{Modos de bajo consumo:} Si tiene mas de un modo de bajo consumo, es mas simple lograr que el sistema se encuentre el mayor tiempo posible consumiendo lo menor posible.
  \item \textbf{Puerto serie:} Interfaces seriales que soporta el micro. Minimamente se necesita que soporten $I^{2}$C y UART.
  \item \textbf{Dimension:} Dimension del micro. El tama\~no de la placa deberia ser lo menor posible por lo que mientras mas peque\~no el micro, mejor.
  \item \textbf{Cantidad de pines:} Dependiendo del encapsulado, habra una cantidad de pines. La cantidad puede afectar el tama\~no y la complejidad de la placa.
  \item \textbf{Precio:} Costo en dolares del integrado.
\end{itemize}

\subsection{Selecci\'on}
En el momento, habia en el laboratorio una placa de desarrollo de Silicon Labs con el microcontrolador C8051F352. Este mismo fue considerado dentro de las elecciones posibles, como puede verse en el cuadro \ref{tabla_micros}.

Ademas de la ventaja de tenerlo en el mismo laboratorio, la placa de Silicon Labs tiene la particularidad de tener una buena ganancia maxima (128x) a la entrada del ADC, lo cual lo distingue del resto de los microcontroladores analizados. Ademas de esto, cumple con el resto de los requisitos propuestos por nuestro tutor, por lo que consistia en una buena eleccion.

Habiendo hecho este analisis, se decidio optar por utilizar el microcontrolador C8051F352 de Silicon Labs.




\clearpage


%Silicon Labs C8051F352
\section{Silicon Labs \emph{C8051F352}} % (fold)
\label{sec:silicon_labs_c8051f352}

En esta secci\'on se explicaran brevemente las funcionalidades que se utilizaron del microcontrolador elegido. Para informaci\'on mas detallada referirse al datasheet del mismo. \cite{bib:datasheet}

\subsection{Conversor Anal\'ogico-Digital} % (fold)
\label{sub:conversor_analogico_digital}

El C8051F350 incluye un ADC Sigma-Delta completamente diferencial de 16 bits con capacidad de calibraci\'on interna, con capacidad de 8 mediciones simultaneas en modo singular, y 4 mediciones simultaneas en modo diferencial. Tiene dos filtros separados de decimaci\'on programables con un throughput de hasta 1KHz. Tiene un voltaje de referencia interno de 2.5 V, y admite la administraci\'on de voltajes de referencia externos. Cada canal puede ser amplificado 1, 2, 4, 8, 16, 32, 64, o 128 veces.

\subsubsection{Modos de medici\'on} % (fold)
\label{ssub:modos_de_medicion}

\begin{itemize}
  \item \textbf{Singular}: La entrada es independiente y medida con respecto a tierra.
  \item \textbf{Diferencial}: Dos entradas medidas con respecto una de la otra.
\end{itemize}
% subsubsection modos_de_medicion (end)

\subsubsection{Modos de conversi\'on} % (fold)
\label{ssub:modos_de_conversion}

\begin{itemize}
  \item \textbf{Singular}: Indica al ADC que genere informaci\'on suficiente para producir un resultado luego de un ciclo de conversi\'on. El filtro puede ser \emph{fast-filter} o \emph{SINC3}. En el primero, el resultado esta disponible luego de un solo ciclo de conversi\'on del ADC. En el segundo, espera 3 ciclos antes de que este disponible el resultado.
  \item \textbf{Continua}: El ADC comienza una nueva conversi\'on inmediatamente despu\'es de terminar la anterior. Los filtros funcionan de manera an\'aloga al modo singular.
\end{itemize}
% subsubsection modos_de_conversi\'on (end)
% subsubsection conversor_anal\'ogico_digital (end)

\subsection{Contadores} % (fold)
\label{sub:contadores}

El C8051F350 cuenta con 4 contadores: Timer0, Timer1, Timer2 y Timer3. Timer0 y Timer1 cuentan con 3 modos de operaci\'on:
\begin{itemize}
  \item Contador de 13 bits
  \item Contador de 16 bits
  \item Contador de 8 bits con valor de retorno
\end{itemize}
Timer0 cuenta con un modo mas que es el de separarse en dos contadores independientes de 8 bits.
Timer2 y Timer3 tienen 2 modos:
 \begin{itemize}
   \item Contador de 16 bits con valor de retorno
   \item Dos contadores independientes de 8 bits con valor de retorno
 \end{itemize}

Todos los timers pueden ser clockeados por fuente interna o externa. En el caso de los timers 2 y 3 la fuente externa se divide por 8.
% subsection contadores (end)

\subsection{Interfaz Serial} % (fold)
\label{sub:interfaz_serial}

El C8051F352 cuenta con dos protocolos de transferencia serial de datos: SMBus y UART. \textbf{SMBus} es una interfaz de dos cables bi-direccional. Es compatible con $I^{2}$C, es por esto que se considera que este microcontrolador cuenta con $I^{2}$C, por el hecho de tener SMBus. \textbf{UART} es una interfaz serial as\'incrona full-duplex.
% subsection interfaz_serial (end)

\subsection{Flash} % (fold)
\label{sub:flash}

Incluye una memoria Flash de 8 Kilobytes re programable dentro del chip para almacenamiento de datos no vol\'atil. Es programable mediante interfaz C2 o por software
% subsection flash (end)
% section silicon_labs_c8051f352 (end)

%Diseño de Software
\section{Diseño del software} % (fold)
\label{sec:diseno_del_software}


Con la eleccion del microcontrolador C8051F350, se procedio a la etapa de diseño del sistema. Para ilustrar el diseño, se utilizan diagramas UML.

\subsubsection{Diagrama de bloques} % (fold)
\label{ssub:diagrama_de_bloques}

\begin{figure}[h]
  \centering
  \includegraphics[width=0.80\textwidth, height = 9cm]{Bloques1}
  \caption{Diagrama de Bloques del sistema}\label{fig:bloques1}
\end{figure}

\paragraph{}
El sistema esta compuesto por 6 bloques o modulos separados, manejados por un modulo principal. En la figura \ref{fig:bloques1} se pueden observar estos bloques y la interaccion que tienen entre si. 
\paragraph{}


\begin{itemize}
  \item El \textbf{Bloque Principal} se encarga ejecutar las funciones del resto de los modulos para dar arranque y ejecucion al sistema.
  \item El \textbf{Bloque Conversor Analógico-Digital} principalmente obtiene los datos de los sensores, los procesa, y los envia al modulo principal. Ademas de esto, configura el funcionamiento del ADC segun los parametros dados por el usuario. El usuario puede elegir la cantidad de pines que va a utilizar como entrada segun la cantidad de sensores que quiera medir, puede elegir un nivel de ganancia de amplificacion de la señal antes de la conversion, y puede tambien elegir el modo de obtencion de los datos (diferencial o single-ended).
  \item El \textbf{Bloque Contador} se encarga de obtener los valores en los contadores de eventos.
  \item El \textbf{Bloque de Interfaz de Usuario} en este modulo se levanta la interfaz con la que interactua el usuario para establecer los parametros configurables del sistema.
  \item El \textbf{Bloque Configurador} interactua directamente con el hardware del microcontrolador. Realiza todas las configuraciones necesarias para poder hacer funcionar cada modulo. Es por esto que en el diagrama de bloques se puede ver que este modulo interactua directamente con todos los demas. Inicializa todos los registros pertinentes, el clock del sistema y setea los puertos de entrada y salida.
  \item El \textbf{Bloque Serial} envia los datos por interfaz UART.
  %\item El \textbf{Bloque Flash} Maneja la unidad de memoria no volatil del sistema. Se utiliza para guardar y cargar configuraciones hechas por el usuario, de forma que puedan cargarse automaticamente al inicio del sistema sin necesidad de volver a configurarlo cada vez que se inicia.
\end{itemize}

\subsection{Mencion sobre la memoria flash} % (fold)
\label{sub:mencion_sobre_la_memoria_flash}

Con respecto al informe de las 100 horas, se removio el bloque flash del diseño. En un principio, se penso en la posibilidad de utilizar la memoria flash del microcontrolador para guardar configuraciones, de forma que no sea necesario configurar el sistema desde 0 cada vez que se inicia. Pero esto no pudo ser posible. Las operaciones necesarias para leer y escribir la memoria y el tamaño de la misma hicieron que sea dificil realizar la escritura y la lectura de la misma. Para escribir una palabra en memoria, es necesario borrar toda la pagina en donde se encuentra la palabra para luego reescribirla. Cada pagina de la memoria flash ocupa 512 Bytes.
Tanto el programa que corre en el microprocesador como los datos de configuracion debe guardarse en la misma memoria flash de 8Kb. El programa ocupa alrededor de 7Kb. En el Kb que sobra, es posible guardar las configuraciones, pero lo que lo dificulta es que el metodo de escritura de la flash pone en riesgo la integridad del programa. Teniendo en cuenta esto, se decidio no utilizar la flash para guardar las configuraciones. Cada vez que se inicia el sistema, es necesario que el usuario ingrese las configuraciones desde 0.

% subsection mencion_sobre_la_memoria_flash (end)

\subsection{Mencion sobre SMBus} % (fold)
\label{sub:mencion_sobre_smbus}

En los requerimientos, se especifica que es necesario que se puedan transmitir los datos de las mediciones utilizando protocolo UART tanto con RS232 como con $I^{2}$C. El microcontrolador cuenta con una interfaz serial[\ref{sec:serial}] llamada SMBus, que puede comunicarse con cualquier dispositivo que utilice $I^{2}$C porque es compatible. Sin embargo, fue complicado lograr hacer funcionar esta funcionalidad, y luego de hacer algunas investigaciones, se encontro que suele ser un problema comun que esta interfaz no suela funcionar en estos microcontroladores. Luego de intentar reiteradas veces sin obtener resultados, se opto por no utilizar SMBus y trabajar unicamente con el protocolo UART.

% subsection mencion_sobre_smbus (end)

% subsubsection diagrama_de_bloques (end)

\subsubsection{Diagramas de caso de uso} % (fold)
\label{ssub:diagramas_de_caso_de_uso}

En la figura \ref{fig:casouso1} se muestra un diagrama de caso de uso del sistema. Los casos de uso son bastante intuitivos, el usuario debe poder configurar todos los aspectos claves del sistema. Este diagrama muestra a grandes rasgos las acciones posibles que pueden hacerse sobre el sistema desde el punto de vista del usuario.

\begin{figure}[h]
  \centering
  \includegraphics[width=0.80\textwidth, height = 9cm]{CasoUso1}
  \caption{Diagrama de caso de uso}\label{fig:casouso1}
\end{figure}

% subsubsection diagramas_de_caso_de_uso (end)

\subsubsection{Diagramas de secuencia} % (fold)
\label{ssub:diagramas_de_secuencia}

Los diagramas de secuencia modelan distintas interacciones entre los componentes de un sistema. En este caso, los dos componentes mas importantes del sistema son el usuario, y el programa principal que recibe las peticiones del usuario a traves de la interfaz grafica, y procesa los pedidos llamando a funciones de otros bloques del sistema. En el primer diagrama (figura \ref{fig:secuencia1}) se modelo una configuracion de un pin especifico para medir una entrada analogica. El programa principal, en este caso, debe llamar a funciones del bloque conversor para configurar el pedido del usuario. Luego, el usuario habilita el comienzo de conversion para que el sistema envie los datos ya digitales por interfaz serial.


\begin{figure}[h]
  \centering
  \includegraphics[width=0.50\textwidth, height = 7cm]{Secuencia1}
  \caption{Diagrama de secuencia del usuario configurando el ADC e iniciando las conversiones}\label{fig:secuencia1}
\end{figure}

El diagrama en la figura \ref{fig:secuencia2} muestra una configuracion de un contador. En este sistema, los contadores se inician junto con el arranque mismo del sistema y desde ahi mismo comienzan a contar los eventos que ocurran en el pin que tienen asignado. Es por esto que lo unico que hay que hacer es consultar el valor en los registros asociados al contador para obtener la cuenta actual.

% El diagrama \ref{fig:secuencia3} muestra una obtencion y un guardado de datos de configuracion en la memoria flash del microcontrolador. Los datos de configuracion son unicamente los pertenecientes al conversor analogico-digital.

\begin{figure}[h]
  \centering
  \includegraphics[width=0.50\textwidth, height = 7cm]{Secuencia2}
  \caption{Diagrama de secuencia del usuario obteniendo la cuenta de uno de los contadores}\label{fig:secuencia2}
\end{figure}


% \begin{figure}[H]
%   \centering
%   \includegraphics[width=0.60\textwidth, height = 8cm]{Secuencia3}
%   \caption{Diagrama de secuencia 3}\label{fig:secuencia3}
% \end{figure}

% subsubsection diagramas_de_secuencia (end)
% section diseno_del_software (end)

\clearpage


%Diseño de Hardware
%Diseño de Hardware
\section{Diseño de hardware} % (fold)
\label{sec:diseno_de_hardware}




Con LOCURAAAAA el microcontrolador seleccionado, se procedió a la etapa de diseño de hardware. La placa de desarrollo C8051F350[\ref{sec:silicon_labs_c8051f352}] con la que se contó en el laboratorio donde se trabajo, sirvió de modelo para diseñar el circuito esquemático en la placa a desarrollar.

\subsection{Diagrama de Bloques de Hardware} % (fold)
\label{sub:diagrama_de_bloques_de_hardware}

La primera etapa de diseño consiste en diseñar un diagrama de bloques que ilustre a grandes rasgos la organización del circuito. Luego de esto se realiza el diseño esquemático que consiste en llevar cada bloque a nivel de componente electrónico y realizar la interconexión necesaria entre todos los elementos existentes para lograr el funcionamiento buscado. Los diagramas de bloque y esquemáticos se pueden ver en las figuras \ref{fig:bloquesHW} y \ref{fig:esquematico}. Con ayuda del software KiCad (\ref{sub:kicad}), se realizo el diagrama esquemático y la implementación en circuito impreso del esquemático construido.

\begin{figure}[/h]
  \centering
  \includegraphics[width=0.80\textwidth, height = 9cm]{bloquesHW}
  \caption{\small Diagrama de bloques del circuito a realizar}\label{fig:bloquesHW}
\end{figure}

Los bloques en la figura \ref{fig:bloquesHW} representan de forma general los distintos módulos a implementar. Las entradas analógicas entran al sistema a través de filtros que reducen el ruido. Luego ingresan al microcontrolador para ser procesadas. El bloque de GPIO y contadores de eventos es simplemente un grupo de pines direccionados a distintas entradas del microcontrolador. GPIO significa ``General Purpose Input Output''( en español, ``entrada y salida de propósito general'' ). Son 4 pines que se separaron para uso general, por necesidad eventual de necesitarlos. Parte de estos GPIO son los pines contadores de eventos, por lo cual se incluyeron dentro del mismo bloque.

Es posible alimentar el sistema por medio del programador, o mediante una tension de 3,3V que se obtiene como salida de un regulador de tension, cuya entrada es de 5V. La llave selectora decide si se alimenta el sistema usando la alimentación por fuente continua de 5V, o por el programador.

% subsection diagrama_de_bloques_de_hardware (end)



\subsection{Diagrama Esquemático} % (fold)
\label{sub:diagrama_esquematico}

La figura \ref{fig:esquematico} muestra el diagrama esquemático del circuito a implementar. El microcontrolador[\ref{sec:silicon_labs_c8051f352}] con el que se trabaja tiene ciertos niveles de tension de operación con el que se trabaja. En principio, se alimenta con una fuente de 5V y 500mA que trabaja junto con un regulador de tension[\cite{bib:lm2937}] que lleve la alimentación a un nivel de 3,3V. Estos 3,3V se utilizan para alimentar al microcontrolador; la tension analógica positiva del conversor; a un integrado MAX232[\cite{bib:max232}] que se utiliza para lograr una interfaz serial entre la UART[\ref{sub:interfaz_serial}] de la placa y un puerto serial de una PC; y a un led testigo de alimentación.

\begin{figure}[/h]
  \centering
  \includegraphics[width=0.95\textwidth, height = 10cm]{esquematico}
  \caption{\small Diagrama esquemático del circuito realizado}\label{fig:esquematico}
\end{figure}


% subsection diagrama_esquemático (end)

\subsection{Implementación en Circuito Impreso} % (fold)
\label{sub:implementacion_en_circuito_impreso}

Una implementación en circuito impreso consiste simplemente en pasar de un diagrama esquemático al despliegue físico de los componentes reales en una PCB (en inglés, Printed Circuit Board). Para esto, se utilizaron funcionalidades del software KiCad, que también se utilizo para realizar el esquemático del circuito. Una imagen del resultado del despliegue de componentes esta en la figura \ref{fig:PCB1}.

\begin{figure}[/h]
  \centering
  \includegraphics[width=0.95\textwidth, height = 10cm]{PCB1}
  \caption{\small Diagrama del despliegue de componentes}\label{fig:PCB1}
\end{figure}

% subsection implementación_en_circuito_impreso (end)

\subsection{Circuito anexo de programación} % (fold)
\label{sub:circuito_anexo_de_programacion}

Para programar el microcontrolador, se utiliza un programador de la marca de Silicon Labs hecho para la placa de desarrollo utilizada. Es posible usar este mismo programador en la plataforma a construir, pero es necesario armar un circuito que funcione como interfaz entre la placa del microcontrolador y el programador de la misma. El circuito se muestra en la figura \ref{fig:PCB2}.

\begin{figure}[/h]
  \centering
  \includegraphics[width=0.20\textwidth, height = 3cm]{PCB2}
  \caption{\small Diagrama del despliegue de componentes del circuito que hace de interfaz entre el programador y la plataforma}\label{fig:PCB2}
\end{figure}

% subsection circuito_anexo_de_programación (end)

\subsection{Impresión del circuito en placa de cobre} % (fold)
\label{sub:impresion_del_circuito_en_placa_de_cobre}

La impresión de la placa física se implemento mas de una vez, hasta que fue posible correr un programa en el sistema y verificar su funcionamiento. Hubo cuatro versiones en total. En cada una de ellas salieron a la vista distintos errores que hacian necesaria una nueva impresion para corregirlos.

\begin{figure}[/h]
  \centering
  \includegraphics[width=0.95\textwidth, height = 10cm]{placa1}
  \caption{\small Fotografia del primer prototipo}\label{fig:placa1}
\end{figure}

La primera version de la placa se puede ver en la figura \ref{fig:placa1}. En esta primera placa hubo errores de diseño del despliegue de componentes, ya que varias de las pistas se solapaban, y esto hubiese provocado cortocircuito entre las pistas.

\begin{figure}[/h]
  \centering
  \includegraphics[width=0.95\textwidth, height = 10cm]{placa2}
  \caption{\small Fotografia del segundo prototipo}\label{fig:placa2}
\end{figure}

En la segunda version se experimentaron problemas relacionados al ancho de los huecos de las patas de algunos componentes. En particular, el encapsulado del regulador de tension de 3,3V, encargado de suministrar tension al microcontrolador, no podia colocarse correctamente en la placa. Al ser una pieza tan importante, dado que sin el el microcontrolador no enciende, hubo que rehacer la placa[\cite{bib:lm2937}]

\begin{figure}[/h]
  \centering
  \includegraphics[width=0.95\textwidth, height = 10cm, angle =90 ]{placa3}
  \caption{\small Fotografia del tercer prototipo}\label{fig:placa3}
\end{figure}



\begin{figure}[/h]
  \centering
  \includegraphics[width=0.95\textwidth, height = 10cm, angle =90 ]{placa4}
  \caption{\small Fotografia del cuarto y ultimo prototipo}\label{fig:placa4}
\end{figure}



% subsection impresión_del_circuito_en_placa_de_cobre (end)

% section diseño_de_hardware (end)

%Estructura de Software

\section{Estructura del software} % (fold)
\label{sec:estructura_del_software}

El programa esta dividido en 6 modulos que en la mayoria de los casos corresponde a cada bloque en el que esta dividido el sistema, como se muestra en la figura \ref{fig:bloques1}

\subsection{Partes que conforman el software} % (fold)
\label{sub:partes_que_conforman_el_software}

En este momento, el programa entero esta compuesto por 6 modulos. Un modulo principal y otros 5 que definen funciones segun los bloques que describen el comportamiento del sistema. El modulo principal ejecuta las funciones de los modulos secundarios.

\paragraph{}
\textbf{Modulos:}
\begin{itemize}
  \item Principal
  \item Conversor
  \item Contador
  \item Interrupciones
  \item Configurador
  \item Interfaz de Usuario
\end{itemize}

% subsection partes_que_conforman_el_software (end)

\subsubsection{Configurador} % (fold)
\label{ssub:configurador}

El configurador se encarga de inicializar todos los parametros necesarios que permiten la operatividad del resto de los modulos. Establece los valores correspondientes para todos los registros pertinentes y configura los puertos seleccionados en el modulo del conversor (Seccin \ref{ssub:conversor}) cuando se especifique por parte del usuario.
\subsubsection{Principal}
El modulo principal es la funcion main, inicializa todo el sistema usando funciones del configurador (Seccion \ref{ssub:configurador}) y hace el loop infinito que corre el sistema indefinidamente, obteniendo los comandos del usuario via la interfaz (Seccion \ref{ssub:interfaz_de_usuario}).

% subsubsection configurador (end)

\subsubsection{Conversor} % (fold)
\label{ssub:conversor}

El modulo de conversion se encarga de la etapa de obtencion y procesado de las se\~nales convertidas del ADC (Secci\'on \ref{sec:adc}).

Las funciones de este modulo se encargan de que la obtencion de los datos se corresponda con la configuracion dada por el usuario. Es por esto que tiene funciones que se activan en la etapa de configuracion del sistema que preparan al mismo para una obtencion de datos conforme a los ajustes hechos por el usuario. Estos ajustes se realizan mediante las instrucciones MML mencionadas en la seccion \ref{ssub:interfaz_de_usuario} y explicadas con detalle en el apendice \ref{sec:instrucciones_mml}.

% subsubsection conversor (end)

\subsubsection{Interfaz de Usuario} % (fold)
\label{ssub:interfaz_de_usuario}

En un principio, se comenzo con una interfaz de usuario pensada en forma de un menu principal. La idea era que un pueda acceder a todas las opciones de configuracion del sistema desde este menu, ingresando opciones por teclado y cambiando asi los parametros. La ventaja de este metodo es que los errores por parte del usuario se reducen significativamente, dado que no tiene otra opcion que elegir entre las opciones que muestra el menu. La desventaja es la complejidad que implica hacer un sistema altamente configurable con una interfaz de este tipo. Esta desventaja fue finalmente la que hizo que se cambiara la interfaz por completo, ya que luego de algunas semanas, las opciones de configuracion comenzaron a crecer y se hizo evidente que una interfaz de menu hacia la configuracion muy arduosa para el usuario y muy complicada de cambiar y hacer para el programador.Por lo que se cambio a un metodo con mas libertades para el usuario y menos complejidad para el programador.

La interfaz de usuario actual esta hecha con un lenguaje de especificacion denominado "Man Machine Languaje". MML es un lenguaje de especificacion que se usa tipicamente para estandarizar la interfaz de un sistema para el manejo del mismo desde una consola. Siguendo el paradigma de MML, lo que se hace es definir una serie de comandos que pueden aceptar una serie de argumentos. Con cada comando y sus argumentos se conforma una orden que ejecuta el procesador. De esta manera, se pueden lograr instrucciones simples que cambian distintos aspectos de la configuracion del sistema conforme a las intenciones del usuario. Una descripcion completa de las instrucciones hechas hasta el momento se encuentra en el apendice \ref{sec:instrucciones_mml}. Con esto, es posible configurar todos los aspectos del sistema, sabiendo como operan todas las instrucciones y sus argumentos.

Este esquema de interfaz de usuario esta todavia en proceso. Hasta ahora, las instrucciones descriptas en el apendice \ref{sec:instrucciones_mml} no cubre por completo todas las configuraciones que deberian poder hacerse teniendo en cuenta los requerimientos principales del sistema.

% subsubsection interfaz_de_usuario (end)

\subsubsection{Contador} % (fold)
\label{ssub:contador}

El modulo de contador contiene funciones que devuelven los valores de las cuentas actuales de los contadores en funcionamiento. Por una cuestion de simpleza, los contadores siempre estan activos, ya sea que se usen o no. Actualmente se cuenta con tres contadores de eventos. El microcontrolador tiene cuatro timers (Secci\'on\ref{ssub:contador} pero obligadamente uno de ellos debe ser utilizado por el modulo de la interfaz serial UART (Secci\'on\ref{ssub:interfaz_serial}).

Cada contador utilizable tiene una funcion que simplemente se encarga de obtener el valor de la cuenta en su respectivo timer asociado. Las instrucciones definidas en MML(Seccion \ref{ssub:interfaz_de_usuario}) descriptas en el apencice \ref{sec:instrucciones_mml} incluyen las instrucciones GT0, GT1, y GT2, que se utilizan para la obtencion del valor de la cuenta actual.


% subsubsection contador (end)

\subsubsection{Interrupciones} % (fold)
\label{ssub:interrupciones}

Este archivo define las rutinas de interrupcion para aquellas funcionalidades que se requiere que realicen interrupciones sobre el microcontrolador para condicionar el comportamiento del programa. Actualmente el conversor analogico-digital\ref{ssub:conversor} es el unico modulo que realiza interrupciones.


% subsubsection interrupciones (end)
% section estructura_del_software (end)


%Apendice
%\appendix
%\begin{appendices}
%   \addappheadtotoc
%   \appendixpage

\section{Apendices} % (fold)
\label{sec:apendices}


\appendix
\section{Instrucciones MML} % (fold)
\label{sec:instrucciones_mml}


\subsection{SSE (Set Single Ended)} % (fold)
\label{sub:sse_set_single_ended}


\textbf{Formato:} SSE,[1]

\textbf{Descripci\'on:}
Establece el pin ingresado en modo single ended.

\textbf{Limitaciones:}
\begin{itemize}
  \item Un solo argumento
  \item El argumento es un byte par comprendido entre 0 y 7
\end{itemize}

\textbf{Ejemplo:}

SSE,4: Establece el pin 4 del ADC en modo single ended
% subsection sse_set_single_ended (end)


\subsection{SDI (Set Diferencial)} % (fold)
\label{sub:sdi_set_diferencial}


\textbf{Formato:} SDI,[1]

\textbf{Descripci\'on:}
Establece el pin ingresado y el pin siguiente a ese en numero en modo diferencial

\textbf{limitaciones:}
\begin{itemize}
  \item un solo argumento
  \item el argumento es un byte par comprendido entre 0 y 6
\end{itemize}

\textbf{ejemplo:}

SDI,2: setea los pines 2 y 3 en modo diferencial

% subsection sdi_set_diferencial (end)
\subsection{SGA (Set Ganancia)} % (fold)
\label{sub:sga_set_ganancia}


\textbf{Formato:} SGA,[1]

\textbf{Descripci\'on:}
Establece la ganancia del ADC segun el argumento a la potencia de 2

\textbf{Limitaciones:}
\begin{itemize}
  \item Un solo argumento
  \item El argumento es un byte comprendido entre 0 y 7
\end{itemize}

\textbf{Ejemplo:}

SGA,3: Establece la ganancia en $2^{3} = 8$
% subsection sga_set_ganancia (end)


\subsection{GT0 (Get Timer 0)} % (fold)
\label{sub:gt0_get_timer_0}


\textbf{Formato:} GT0[0]

\textbf{Descripci\'on:}
Obtiene el valor actual de la cuenta de los eventos digitales monitoreados por timer0

\textbf{Limitaciones:}
\begin{itemize}
  \item No lleva argumentos
\end{itemize}

% subsection gt0_get_timer_0_ (end)
\subsection{GT2 (Get Timer 2)} % (fold)
\label{sub:gt2_get_timer_2}

\textbf{Formato:} GT2[0]

\textbf{Descripci\'on:}
Obtiene el valor actual de la cuenta de los eventos digitales monitoreados por timer2

\textbf{Limitaciones:}
\begin{itemize}
  \item No lleva argumentos
\end{itemize}

% subsection gt2_get_timer_2_ (end)
\subsection{ST (Start)} % (fold)
\label{sub:st_start}


\textbf{Formato:} ST[0]

\textbf{Descripci\'on:}
Guarda los cambios, sale del modo de configuracion y comienza a correr el programa

\textbf{Limitaciones:}
\begin{itemize}
  \item No lleva argumentos
\end{itemize}

% subsection st_start_ (end)

\clearpage

% section instrucciones_mml (end)

\section{Componentes Electronicos Utilizados} % (fold)
\label{sec:componentes_electronicos_utilizados}


% \subsection{Optoacoplador T2103} % (fold)
% \label{ssub:optoacoplador_t2103}



% subsubsection optoacoplador_t2103 (end)


\subsection{Regulador de Tension LM2937} % (fold)
\label{sub:regulador_de_tension_lm2937}

Es un regulador de tension positivo capaz de soportar una corriente de carga de 500mA. Desarrollado por Texas Instruments. Ideal para aplicaciones de automatizacion. Tiene proteccion intrinseca contra corrientes inversas, y proteccion contra cortocircuitos. \\

\textbf{Mas informacion:} http://www.ti.com/lit/ds/symlink/lm2937.pdf

% subsection regulador_de_tension_lm2937 (end)

\subsection{Driver MAX232} % (fold)
\label{sub:driver_max232}

Es un driver receptor/emisor que incluye un voltaje capacitivo generador para proveer voltajes a niveles TTL de una alimentacion de 5V. Un receptor convierte entradas de tipo s TIA/EIA-232-F a niveles TTL/CMOS. Tiene una histeresis tipica de 0.5V. \\


\textbf{Mas informacion:} http://www.ti.com/lit/ds/symlink/max232.pdf

% subsection driver_max232 (end)





% section componentes_electronicos_utilizados (end)

\section{Sofware Utilizado} % (fold)
\label{sec:sofware_utilizado}


\subsubsection{uVision 4} % (fold)
\label{ssub:uvision_4}

lalala

% subsubsection uvision_4 (end)


\clearpage


% section sofware_utilizado (end)

\clearpage

% section apendices (end)


%Bibliografia
% Bibliografía.
%-----------------------------------------------------------------
\begin{thebibliography}{99}

\bibitem{bib:datasheet} Silicon Laboratories \emph{C8051F351/2/3} 8K ISP Flash MCU Family
\bibitem{bib:kicad} KiCad EDA: A Cross Platform and Open Source Electronics Design Automation Suite (http://kicad-pcb.org/help/documentation)

\end{thebibliography}







\end{document} 